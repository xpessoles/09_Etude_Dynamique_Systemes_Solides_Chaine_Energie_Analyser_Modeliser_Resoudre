\documentclass[10pt,fleqn]{article} % Default font size and left-justified equations
\usepackage[%
    pdftitle={Dynamique : Inertie équivalente},
    pdfauthor={Xavier Pessoles}]{hyperref}
    
\input{style/new_style}
\input{style/macros_SII}

\usepackage{multicol}
\usepackage{style/schemabloc}
\fichetrue
%\fichefalse

\proftrue
%\proffalse

\tdtrue
%\tdfalse

%\courstrue
\coursfalse

\def\discipline{Sciences \\Industrielles de \\ l'Ingénieur}
\def\xxtete{Sciences Industrielles de l'Ingénieur}

\def\classe{PT -- PT$\star$}
\def\xxnumpartie{Partie n}
\def\xxpartie{Méthode de résolution permettant la détermination les actions mécaniques dans les liaisons ou les lois de déplacement en dynamique\\
Analyse, Modélisation, Résolution}

\def\xxnumchapitre{Chapitre n}
\def\xxchapitre{Titre Chapitre}

\def\xxtitreexo{Exercices d'application}
\def\xxsourceexo{\hspace{.2cm}}


\def\xxposongletx{2}
\def\xxposonglettext{1.45}
\def\xxposonglety{20}
\def\xxonglet{Part. ** -- Ch. **}

\def\xxactivite{Applications}
\def\xxauteur{\textsl{Xavier Pessoles}}

\def\xxcompetences{%
\textsl{%
\textbf{Savoirs et compétences :}\\
\noindent \textbf{Résoudre :} à partir des modèles retenus :
\begin{itemize}[label=\ding{112},font=\color{ocre}] 
\item ***
\item ***
\end{itemize}
\begin{itemize}[label=\ding{112},font=\color{ocre}] 
\item ***
\end{itemize}
%
%\noindent \textit{Mod2 -- C4.1 :} Représentation par schéma bloc.
}}

\def\xxfigures{
\includegraphics[width=.8\textwidth]{images/rdm}
}%figues de la page de garde

\def\xxpied{%
Partie * -- ** -- Analyse, Modélisation, Résolution \\
Ch. * : ** -- \xxactivite%
}


\setcounter{secnumdepth}{5}
%---------------------------------------------------------------------------


\begin{document}
%\chapterimage{png/Fond_Cin}
\input{style/new_pagegarde}
\vspace{10cm}
\pagestyle{fancy}
\thispagestyle{plain}


\def\columnseprulecolor{\color{ocre}}
\setlength{\columnseprule}{0.4pt} 

%\begin{multicols}{2}

\section*{Exercice 1 -- Calcul de l'inertie équivalente d'un train épicycloïdal}
\setcounter{subparagraph}{0}
On considère le train épicycloïdal suivant à trois satellites. On donne l'inertie de chacune des pièces :
$$
\overline{\overline{I_A}}(1/0) = 
\begin{pmatrix} 
A_1 & 0 & 0 \\
0 & B_1 & 0 \\
0 & 0 & C_1 \\
\end{pmatrix}
\quad
\overline{\overline{I_B}}(2/0) = 
\begin{pmatrix} 
A_2 & 0 & 0 \\
0 & B_2 & 0 \\
0 & 0 & C_2 \\
\end{pmatrix}
\quad
\overline{\overline{I_A}}(3/0) = 
\begin{pmatrix} 
A_3 & 0 & 0 \\
0 & B_3 & 0 \\
0 & 0 & C_3 \\
\end{pmatrix}
$$

\begin{center}
\includegraphics[width=.45\textwidth]{images/train_01}
\end{center}

\subparagraph{}
\textit{Déterminer le rapport de réduction du train épicycloïdal.}
\ifprof
\begin{corrige}
\begin{methode}
\begin{enumerate}
\item Écrire le rapport de réduction recherché.
\item Refaire le schéma en fixant le porte satellite et en libérant le bâti. Le porte satellite devient donc le bâti et le train peut être considéra comme un train simple.
\item Déterminer le rapport de réduction du train simple (les taux de rotation seront donc exprimés en fonction du porte-satellite) en fonction du nombre de dents des roues dentées.
\item Introduire les fréquences de rotation exprimées au point 1.
\item Exprimer le rapport de réduction cherché en fonction du  nombre de dents des solides. 
\end{enumerate}
\end{methode}

\begin{minipage}[c]{.6\linewidth}
On recherche $k=\dfrac{\omega(3/0)}{\omega(1/0)}$. 

On bloque le porte satellite \textbf{3} et on libère la couronne \textbf{0}. 

On peut donc exprimer $ \dfrac{\omega(0/3)}{\omega(1/3)} = (-1)^1\dfrac{Z_1\cdot Z_2}{Z_2\cdot Z_0} = -\dfrac{Z_1}{Z_0}$.

En décomposant le taux de rotation, on introduit $\omega(1/0)$ et $\omega(0/3)$ :
$ \dfrac{\omega(0/3)}{\omega(1/3)} =  \dfrac{\omega(0/3)}{\omega(1/0)+\omega(0/3)} = -\dfrac{Z_1}{Z_0} 
\Leftrightarrow \dfrac{-\omega(3/0)}{\omega(1/0)-\omega(3/0)}  =-\dfrac{Z_1}{Z_0}  \Leftrightarrow {Z_0} \omega(3/0)  =Z_1 \left( \omega(1/0)-\omega(3/0)\right) 
\Leftrightarrow  \omega(3/0) \left(Z_0 + Z_1\right) =Z_1 \omega(1/0) 
\Leftrightarrow \dfrac{\omega(3/0)}{\omega(1/0)} = \dfrac{Z_1}{Z_1+Z_0}$.

Au final, $k = \dfrac{\omega(3/0)}{\omega(1/0)} = \dfrac{Z_1}{Z_1+Z_0}$.
\end{minipage}\hfill
\begin{minipage}[c]{.35\linewidth}
\begin{center}
\includegraphics[width=\textwidth]{images/train_02}
\end{center}
\end{minipage}
\end{corrige}
\else
\fi

\subparagraph{}
\textit{Déterminer l'énergie cinétique de l'ensemble $E=\{1,2,3\}$ par rapport au référentiel de la pièce \textbf{0} supposé galiléen.}
\ifprof
\begin{corrige}
\begin{methode}
\begin{enumerate}
\item On calcule $T(1/0)$, 
\end{enumerate}
\end{methode}
\textbf{Calcul de l'énergie cinétique $T(1/0)$}

Par définition, 
$2T(1/0)=\left\{\mathcal{V}(1/0) \right\} \times \left\{\mathcal{C}(1/0) \right\} $
$A$ étant un point fixe dans \textbf{0}, on a : 
$$
\left\{\mathcal{V}(1/0) \right\} = \torseurl{\vecto{1}{0}=\omega(1/0)\vect{z_0}}{\vectv{A}{1}{0}=\vect{0}}{A} \quad 
\left\{\mathcal{C}(1/0) \right\} 
= \torseurl{M_1\vectv{G}{1}{0}}{\vect{\sigma \left( A\in 1/0\right)}=\overline{\overline{I}}\left( A,0\right) \vecto{1}{0}=C_1\omega(1/0)\vect{z}}{A}
$$
\end{corrige}
\else
\fi



\end{document}



\subparagraph{}
\textit{}

\ifprof
\begin{corrige}
\end{corrige}
\else
\fi
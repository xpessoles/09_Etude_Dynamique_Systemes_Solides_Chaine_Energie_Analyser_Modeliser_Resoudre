\documentclass[10pt,fleqn]{article} % Default font size and left-justified equations
\usepackage[%
    pdftitle={Dynamique : Inertie équivalente},
    pdfauthor={Xavier Pessoles}]{hyperref}
    
\input{style/new_style}
\input{style/macros_SII}

\usepackage{multicol}
\usepackage{style/schemabloc}
\fichetrue
%\fichefalse

\proftrue
%\proffalse

\tdtrue
%\tdfalse

%\courstrue
\coursfalse

\def\discipline{Sciences \\Industrielles de \\ l'Ingénieur}
\def\xxtete{Sciences Industrielles de l'Ingénieur}

\def\classe{PT -- PT$\star$}
\def\xxnumpartie{Partie n}
\def\xxpartie{Méthode de résolution permettant la détermination les actions mécaniques dans les liaisons ou les lois de déplacement en dynamique\\
Analyse, Modélisation, Résolution}

\def\xxnumchapitre{Chapitre n}
\def\xxchapitre{Titre Chapitre}

\def\xxtitreexo{Exercices d'application}
\def\xxsourceexo{\hspace{.2cm}}


\def\xxposongletx{2}
\def\xxposonglettext{1.45}
\def\xxposonglety{20}
\def\xxonglet{Part. ** -- Ch. **}

\def\xxactivite{Applications}
\def\xxauteur{\textsl{Xavier Pessoles}}

\def\xxcompetences{%
\textsl{%
\textbf{Savoirs et compétences :}\\
\noindent \textbf{Résoudre :} à partir des modèles retenus :
\begin{itemize}[label=\ding{112},font=\color{ocre}] 
\item ***
\item ***
\end{itemize}
\begin{itemize}[label=\ding{112},font=\color{ocre}] 
\item ***
\end{itemize}
%
%\noindent \textit{Mod2 -- C4.1 :} Représentation par schéma bloc.
}}

\def\xxfigures{
\includegraphics[width=.8\textwidth]{images/rdm}
}%figues de la page de garde

\def\xxpied{%
Partie * -- ** -- Analyse, Modélisation, Résolution \\
Ch. * : ** -- \xxactivite%
}


\setcounter{secnumdepth}{5}
%---------------------------------------------------------------------------


\begin{document}
%\chapterimage{png/Fond_Cin}
\input{style/new_pagegarde}
\vspace{10cm}
\pagestyle{fancy}
\thispagestyle{plain}


\def\columnseprulecolor{\color{ocre}}
\setlength{\columnseprule}{0.4pt} 

%\begin{multicols}{2}

\section*{Exercice 1 -- Calcul de l'inertie équivalente d'un train épicycloïdal}
\setcounter{subparagraph}{0}
On donne le schéma ci-dessous.
\begin{center}
\includegraphics[width=.45\textwidth]{images/fig_01}
\end{center}

\subparagraph{}
\textit{Déterminer le degré d'hyperstatisme.}

\subparagraph{}
\textit{Déterminer les actions mécaniques en $A$ et en $B$ en fonction d'une seule inconnue.}


\ifprof
\begin{corrige}
En isolant la poutre, en réalisant un bilan des actions mécaniques et en réalisant le PFS en $A$, on obtient : 
$
\left\{
\begin{array}{l}
X_A  = 0 \\
Y_A  +Y_B - F= 0 \\
N_A-Fa+Y_B L = 0 \\
\end{array}
\right.
\Leftrightarrow
\left\{
\begin{array}{l}
X_A = 0 \\
Y_A = F-Y_B \\
Y_B = \dfrac{Fa - N_A}{L} \\
\end{array}
\right.
$

\end{corrige}
\else
\fi



\subparagraph{}
\textit{Après avoir identifier les différentes parties constituant la poutre, déterminer les torseurs de cohésion sur chacune de ces parties.}

\ifprof
\begin{corrige}

\textbf{On étudie tout d'abord la partie $[AP]$.}
\begin{center}
\includegraphics[width=.45\textwidth]{images/fig_02}
\end{center}

%On isole la partie $II$ et on applique le PFS :
%$$
%\torseurscoh_{I \rightarrow II} + \left\{\mathcal{T}_{\text{F}\rightarrow II}\right\}  = \{0\}
%\Longleftrightarrow \torseurscoh_{II \rightarrow I} = \left\{\mathcal{T}_{\text{F}\rightarrow II}\right\}  
%$$
%
%Détermination de $ \left\{\mathcal{T}_{\text{F}\rightarrow II}\right\}$ :
%
%$\left\{\mathcal{T}_{\text{F}\rightarrow II}\right\} 
%= \torseurl{\vectf{\text{F}}{II}=-F\vect{y}}{\vectm{G}{\text{F}}{II}=-(a-x) F \vect{z}}{G}$
%
%Détermination de $ \left\{\mathcal{T}_{\text{II}\rightarrow I}\right\}$ :
%
%$
%\forall x \in [0,a] : \quad 
%\left\{\mathcal{T}_{\text{II}\rightarrow I}\right\} 
%= \torseurcol{N}{T_y}{T_z}{M_t}{M_{fy}}{M_{fz}}{G}
%=  \torseurcol{0}{-F}{0}{0}{0}{-(a-x) F }{G}
%$

En isolant la partie $I$ et en appliquant le PFS, on a : 
$$
\torseurscoh_{II \rightarrow I} + \left\{\mathcal{T}_{\text{ext}\rightarrow I}\right\}  = \{0\}
\Longleftrightarrow \torseurscoh_{II \rightarrow I} =  - \left\{\mathcal{T}_{\text{ext}\rightarrow I}\right\}
$$ 

Détermination de $\left\{\mathcal{T}_{\text{ext}\rightarrow I}\right\}$ :

$\left\{\mathcal{T}_{\text{ext}\rightarrow I}\right\}
=  \torseurcol{0}{Y_A}{0}{0}{0}{N_A}{A} 
=  \torseurcol{0}{Y_A}{0}{0}{0}{N_A-xY_A}{G} 
$ 


Détermination de $ \left\{\mathcal{T}_{\text{II}\rightarrow I}\right\}$ :

$
\forall x \in [0,a] : \quad 
\left\{\mathcal{T}_{\text{II}\rightarrow I}\right\} 
= \torseurcol{N}{T_y}{T_z}{M_t}{M_{fy}}{M_{fz}}{G}
=  \torseurcol{0}{-Y_A}{0}{0}{0}{xY_A-N_A}{G} 
$

\end{corrige}

\begin{corrige}

\textbf{On étudie ensuite la partie $[PB]$.}
\begin{center}
\includegraphics[width=.45\textwidth]{images/fig_03}
\end{center}

On isole la partie II, on réalise le bilan des actions mécaniques extérieures et on applique le PFS : 

$$
\torseurscoh_{I \rightarrow II} + \left\{\mathcal{T}_{\text{Ext}\rightarrow II}\right\}  = \{0\}
\Longleftrightarrow \torseurscoh_{II \rightarrow I} = \left\{\mathcal{T}_{\text{Ext}\rightarrow II}\right\}  
$$


Détermination de $\left\{\mathcal{T}_{\text{ext}\rightarrow II}\right\}$ :


$\left\{\mathcal{T}_{\text{ext}\rightarrow II}\right\}
=  \torseurcol{0}{Y_B}{0}{0}{0}{0}{B} 
=  \torseurcol{0}{Y_B}{0}{0}{0}{(L-x) Y_B}{G} 
$ 



Détermination de $ \left\{\mathcal{T}_{\text{II}\rightarrow I}\right\}$ :

$
\forall x \in [a,L] : \quad 
\left\{\mathcal{T}_{\text{II}\rightarrow I}\right\} 
= \torseurcol{N}{T_y}{T_z}{M_t}{M_{fy}}{M_{fz}}{G}
=  \torseurcol{0}{Y_B}{0}{0}{0}{(L-x) Y_B}{G} 
$

\end{corrige}



\else
\fi

%\end{multicols}


\ifprof
\begin{corrige}
Équation de la déformée sur la première partie : 
%$
%\forall x \in [0,a] : \quad EI_{Gz} y_1''(x)= M_{fz} \Leftrightarrow EI_{Gz} y_1''(x)= -(a-x)F
%\Leftrightarrow EI_{Gz} y_1''(x)= xF-aF$
%
%$ \Rightarrow EI_{Gz} y_1'(x)= \dfrac{1}{2}Fx^2-aFx + v_1$
%
%$\Rightarrow EI_{Gz} y_1(x)= \dfrac{1}{6}Fx^3-\dfrac{1}{2}aFx^2 + v_1x + y_1$
%
%
%La poutre étant encastrée en $A$, on a donc $y_1(0)=0$ et $y_1'(0)=0$. 
%En conséquences, $y_1=0$ et $v_1 = 0$ et  
%$$EI_{Gz}y_1(x)= \dfrac{1}{6}Fx^3-\dfrac{1}{2}aFx^2 $$

$
\forall x \in [0,a] : \quad EI_{Gz} y_1''(x)= M_{fz} \Leftrightarrow EI_{Gz} y_1''(x)= xY_A - N_A$

$ \Rightarrow EI_{Gz} y_1'(x)= \dfrac{1}{2}x^2Y_A - N_Ax + v_1$

$\Rightarrow EI_{Gz} y_1(x)= \dfrac{1}{6}x^3Y_A - \dfrac{1}{2}N_Ax^2 + v_1x + y_1$


La poutre étant encastrée en $A$, on a donc $y_1(0)=0$ et $y_1'(0)=0$. 
En conséquences, $y_1=0$ et $v_1 = 0$ et  
$$EI_{Gz}y_1(x)= \dfrac{1}{6}Y_A x^3 - \dfrac{1}{2}N_Ax^2 $$


Équation de la déformée sur la seconde partie : 
$
\forall x \in [a,L] : \quad EI_{Gz} y_2''(x)= M_{fz} \Leftrightarrow EI_{Gz} y_2''(x)= (L-x)Y_B
\Leftrightarrow EI_{Gz} y_2''(x)= -xY_B+LY_B$

$ \Rightarrow EI_{Gz} y_2'(x)= -\dfrac{1}{2}Y_B x^2+LY_Bx + v_2$

$\Rightarrow EI_{Gz} y_2(x)= -\dfrac{1}{6}Y_B x^3+\dfrac{1}{2}LY_Bx^2 + v_2x+y_2$


La poutre étant en appui ponctuel en $B$, on a donc $y_2(L)=0$. 
En conséquences,   
$$y_2= \dfrac{1}{6}Y_B L^3-\dfrac{1}{2}Y_BL^3 - v_2L = -\dfrac{1}{3}Y_B L^3- v_2L $$


La poutre étant un solide continu, on a donc nécessairement $y_1(a)=y_2(a)$ et $y'_1(a)=y'_2(a)$.
D'où :

$$
\left\{
\begin{array}{l}
\dfrac{1}{6}Y_A a^3 - \dfrac{1}{2}N_Aa^2   = -\dfrac{1}{6}Y_B a^3+\dfrac{1}{2}LY_Ba^2 + v_2a  -\dfrac{1}{3}Y_B L^3- v_2L  \\
\dfrac{1}{2}a^2Y_A - N_Aa   = -\dfrac{1}{2}Y_B a^2+LY_Ba + v_2
\end{array}
\right.
$$


$$
\Longleftrightarrow
\left\{
\begin{array}{l}
Y_A a^3 - 3N_Aa^2   = -Y_B a^3+3LY_Ba^2 + 6v_2a  -2Y_B L^3- 6v_2L  \\
v_2 = 
\end{array}
\right.
$$
$$
Y_A a^3 - 3N_Aa^2   = -Y_B a^3+3LY_Ba^2 + 6a\left( \dfrac{1}{2}a^2Y_A - N_Aa  +\dfrac{1}{2}Y_B a^2- LY_Ba \right)  -2Y_B L^3- 6L\left( \dfrac{1}{2}a^2Y_A - N_Aa  +\dfrac{1}{2}Y_B a^2- LY_Ba \right)
$$

$$
Y_A a^3 - 3N_Aa^2   = -Y_B a^3+3LY_Ba^2 +  3a^3Y_A - 6aN_Aa  +3Y_B a^3- 6aLY_Ba   -2Y_B L^3-3 La^2Y_A + 6LN_Aa  -3L Y_B a^2-+ 6L^2 aY_B 
$$
$$
0  = 2Y_B a^3 +  2a^3Y_A - 3N_Aa^2 - 6LY_Ba^2   -2Y_B L^3-3 La^2Y_A + 6LN_Aa  - 6L^2 aY_B  
$$
$$
\Longleftrightarrow
 N_A \left(   3a^2- 6La \right)  = Y_B\left(2 a^3  - 6La^2   -2 L^3 - 6L^2 a\right) +  Y_A\left(2a^3  -3 La^2\right)    
$$
On a donc : 
$$
\left\{
\begin{array}{l}
 N_A   = Y_B\dfrac{\left(2 a^3  - 6La^2   -2 L^3 - 6L^2 a\right)}{\left(   3a^2- 6La \right)} +  Y_A\dfrac{\left(2a^3  -3 La^2\right)}{\left(   3a^2- 6La \right)}   \\
Y_A = F-Y_B \\
N_A = Fa -  Y_B L \\
\end{array}
\right.
$$

$$
\Leftrightarrow 
\left\{
\begin{array}{l}
 N_A   = Y_B\dfrac{\left(2 a^3  - 6La^2   -2 L^3 - 6L^2 a\right)}{\left(   3a^2- 6La \right)} +  Y_A\dfrac{\left(2a^3  -3 La^2\right)}{\left(   3a^2- 6La \right)}   \\
Y_A = F-Y_B \\
Y_B\left(2 a^3  - 6La^2   -2 L^3 - 6L^2 a\right)+  Y_A\left(2a^3  -3 La^2\right) = Fa\left(   3a^2- 6La \right) -  Y_B L\left(   3a^2- 6La \right) \\
\end{array}
\right.
$$

$$
\Leftrightarrow 
\left\{
\begin{array}{l}
 N_A   = Y_B\dfrac{\left(2 a^3  - 6La^2   -2 L^3 - 6L^2 a\right)}{\left(   3a^2- 6La \right)} +  Y_A\dfrac{\left(2a^3  -3 La^2\right)}{\left(   3a^2- 6La \right)}   \\
Y_A = F-Y_B \\
2 Y_Ba^3  - 6Y_BLa^2   -2 Y_BL^3 - 6Y_BL^2 a+   2Fa^3  -3F La^2 -Y_B 2a^3  +Y_B3 La^2 =    3Fa^3- F6La^2 -  Y_B L   3a^2+  Y_B 6L^2a  \\
\end{array}
\right.
$$

$$
\Leftrightarrow 
\left\{
\begin{array}{l}
 N_A   = Y_B\dfrac{\left(2 a^3  - 6La^2   -2 L^3 - 6L^2 a\right)}{\left(   3a^2- 6La \right)} +  Y_A\dfrac{\left(2a^3  -3 La^2\right)}{\left(   3a^2- 6La \right)}   \\
Y_A = F-Y_B \\
   Y_B \left( -2 L^3 - 12L^2 a\right)     =    Fa^3- 3FLa^2   \\
\end{array}
\right.
$$

$$
\Leftrightarrow 
\left\{
\begin{array}{l}
 N_A   = Y_B\dfrac{\left(2 a^3  - 6La^2   -2 L^3 - 6L^2 a\right)}{\left(   3a^2- 6La \right)} +  Y_A\dfrac{\left(2a^3  -3 La^2\right)}{\left(   3a^2- 6La \right)}   \\
Y_A = F-Y_B \\
   Y_B    =    Fa^2\dfrac{3L-a}{ 2L^2\left( L + 6 a\right) }   \\
\end{array}
\right.
$$
\end{corrige}
\else
\fi

\section*{Exercice 2 : Déformation d'une poutre bi-encastrée}

\begin{center}
\includegraphics[width=.48\linewidth]{images/fig_04}
\end{center}

\setcounter{subparagraph}{0}
\subparagraph{}
\textit{Déterminer la flèche maximale.}
\ifprof
\begin{corrige}
~\\
On résout le problème en raisonnant par symétrie. On cherche donc à résoudre le problème suivant.
\begin{center}
\includegraphics[width=.48\linewidth]{images/fig_05}
\end{center}

On isole la poutre, on réalise le bilan des actions mécaniques et on exprime le PFS au point $A$ :
$$
\left\{\mathcal{T}_{\text{Ext} \rightarrow \text{Poutre}}  \right\} + 
\left\{\mathcal{T}_{\text{AP} \rightarrow \text{Poutre}}  \right\} =
\left\{0 \right\} 
$$ 


$$
\torseurcol{X_A}{Y_A}{0}{0}{0}{N_A}{A} + 
\torseurcol{X_P}{0}{0}{0}{0}{N_P}{P} + 
\torseurcol{0}{-\dfrac{F}{2}}{0}{0}{0}{0}{P} =  
\torseurcol{0}{0}{0}{0}{0}{0}{A}
$$


$$
\torseurcol{X_A}{Y_A}{0}{0}{0}{N_A}{A} + 
\torseurcol{X_P}{0}{0}{0}{0}{N_P}{A} + 
\torseurcol{0}{-\dfrac{F}{2}}{0}{0}{0}{-\dfrac{F}{2}\dfrac{L}{2}}{A} =  
\torseurcol{0}{0}{0}{0}{0}{0}{A}
$$
Au final, 
$$
\left\{
\begin{array}{l}
X_A +X_P= 0\\
Y_A - \dfrac{F}{2}= 0 \\
N_A + N_P -\dfrac{FL}{4} = 0
\end{array}
\right.
$$


Détermination de $ \left\{\mathcal{T}_{\text{II}\rightarrow I}\right\}$ :

$
\forall x \in [0,L/2] : \quad 
\left\{\mathcal{T}_{\text{II}\rightarrow I}\right\} 
= \torseurcol{N}{T_y}{T_z}{M_t}{M_{fy}}{M_{fz}}{G}
= -  \torseurcol{X_A}{\dfrac{F}{2}}{0}{0}{0}{N_A-x\dfrac{F}{2}}{G} 
$


La déformée de la poutre est donnée par : 
$$
EI_{Gz}y''(x) = M_{fz}(x) 
\Longleftrightarrow EI_{Gz}y''(x) = -N_A+x\dfrac{F}{2}
\Rightarrow EI_{Gz}y'(x) = -N_Ax+\dfrac{F}{4}x^2 + v_0
$$

$$ \Rightarrow EI_{Gz}y(x) = -\dfrac{1}{2}N_Ax^2+\dfrac{F}{12}x^3 + v_0x + y_0 $$

La poutre étant en liaison encastrement en $A$ on a : $y(0)=0$ et $y'(0)=0$.
En conséquences, $v_0=0$ et $y_0=0$.

Par continuité de la matière, $y'(L/2)=0$. En conséquences, 
$$ 0 =  -N_A \dfrac{L}{2}+\dfrac{F}{4}\dfrac{L^2}{4}
\Leftrightarrow N_A  = \dfrac{FL}{8}
 $$

Au final, 
$$ 
 EI_{Gz}y(x) = -  \dfrac{FL}{16} x^2+\dfrac{F}{12}x^3 
$$

La flèche est obtenue lorsque $x = L/2$ en conséquences, 
$$
f = \dfrac{1}{ EI_{Gz}}\left[  - \dfrac{FL}{16} x^2+\dfrac{F}{12}x^3 \right]
\Leftrightarrow 
f = \dfrac{FL^3}{ EI_{Gz}}\left[  - \dfrac{1}{64} +\dfrac{1}{96} \right] =  - \dfrac{FL^3}{192 EI_{Gz}}
$$


$$
$$
\end{corrige}
\else
\fi

\end{document}



\subparagraph{}
\textit{}

\ifprof
\begin{corrige}
\end{corrige}
\else
\fi